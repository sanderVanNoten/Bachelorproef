%===============================================================================
% LaTeX sjabloon voor de bachelorproef toegepaste informatica aan HOGENT
% Meer info op https://github.com/HoGentTIN/latex-hogent-report
%===============================================================================

\documentclass[dutch,dit,thesis]{hogentreport}

% TODO:
% - If necessary, replace the option `dit`' with your own department!
%   Valid entries are dbo, dbt, dgz, dit, dlo, dog, dsa, soa
% - If you write your thesis in English (remark: only possible after getting
%   explicit approval!), remove the option "dutch," or replace with "english".

\usepackage{lipsum} % For blind text, can be removed after adding actual content
\usepackage{listings}
\usepackage{xcolor}
\usepackage{tabularx}
\usepackage{}
\definecolor{codegreen}{rgb}{0,0.6,0}
\definecolor{codegray}{rgb}{0.5,0.5,0.5}
\definecolor{codepurple}{rgb}{0.58,0,0.82}
\definecolor{backcolour}{rgb}{0.95,0.95,0.92}

\lstdefinestyle{mystyle}{
    backgroundcolor=\color{backcolour},   
    commentstyle=\color{codegreen},
    keywordstyle=\color{magenta},
    numberstyle=\tiny\color{codegray},
    stringstyle=\color{codepurple},
    basicstyle=\ttfamily\footnotesize,
    breakatwhitespace=false,         
    breaklines=true,                 
    captionpos=b,                    
    keepspaces=true,                 
    numbers=left,                    
    numbersep=5pt,                  
    showspaces=false,                
    showstringspaces=false,
    showtabs=false,                  
    tabsize=2
}

\lstset{style=mystyle}
%% Pictures to include in the text can be put in the graphics/ folder
\graphicspath{{graphics/}}

%% For source code highlighting, requires pygments to be installed
%% Compile with the -shell-escape flag!
\usepackage[section]{minted}
\usemintedstyle{solarized-light}
\definecolor{bg}{RGB}{253,246,227} %% Set the background color of the codeframe

%% Change this line to edit the line numbering style:
\renewcommand{\theFancyVerbLine}{\ttfamily\scriptsize\arabic{FancyVerbLine}}

%% Macro definition to load external java source files with \javacode{filename}:
\newmintedfile[javacode]{java}{
    bgcolor=bg,
    fontfamily=tt,
    linenos=true,
    numberblanklines=true,
    numbersep=5pt,
    gobble=0,
    framesep=2mm,
    funcnamehighlighting=true,
    tabsize=4,
    obeytabs=false,
    breaklines=true,
    mathescape=false
    samepage=false,
    showspaces=false,
    showtabs =false,
    texcl=false,
}

% Other packages not already included can be imported here

%%---------- Document metadata -------------------------------------------------
% TODO: Replace this with your own information
\author{Sander Van Noten}
\supervisor{Mevr. F. Spriet}
\cosupervisor{Mr. A. Seys}
\title[]%
    {Toolset voor automatisatie en controle van Fortigate firewall regels in Azure}
\academicyear{\advance\year by -1 \the\year--\advance\year by 1 \the\year}
\examperiod{2}
\degreesought{\IfLanguageName{dutch}{Professionele bachelor in de Toegepaste Informatica}{Bachelor of applied Computer Science}}
\partialthesis{false} %% To display 'in partial fulfilment'
%\institution{Internshipcompany BVBA.}

%% Add global exceptions to the hyphenation here
\hyphenation{back-slash}

%% The bibliography (style and settings are  found in hogentthesis.cls)
\addbibresource{bachproef.bib}            %% Bibliography file
\addbibresource{../voorstel/voorstel.bib} %% Bibliography research proposal
\defbibheading{bibempty}{}

%% Prevent empty pages for right-handed chapter starts in twoside mode
\renewcommand{\cleardoublepage}{\clearpage}

\renewcommand{\arraystretch}{1.2}

%% Content starts here.
\begin{document}

%---------- Front matter -------------------------------------------------------

\frontmatter

\hypersetup{pageanchor=false} %% Disable page numbering references
%% Render a Dutch outer title page if the main language is English
\IfLanguageName{english}{%
    %% If necessary, information can be changed here
    \degreesought{Professionele Bachelor toegepaste informatica}%
    \begin{otherlanguage}{dutch}%
       \maketitle%
    \end{otherlanguage}%
}{}

%% Generates title page content
\maketitle
\hypersetup{pageanchor=true}

%%=============================================================================
%% Voorwoord
%%=============================================================================

\chapter*{\IfLanguageName{dutch}{Woord vooraf}{Preface}}%
\label{ch:voorwoord}


%% TODO:
%% Het voorwoord is het enige deel van de bachelorproef waar je vanuit je
%% eigen standpunt (``ik-vorm'') mag schrijven. Je kan hier bv. motiveren
%% waarom jij het onderwerp wil bespreken.
%% Vergeet ook niet te bedanken wie je geholpen/gesteund/... heeft


\input{samenvatting}

%---------- Inhoud, lijst figuren, ... -----------------------------------------

\tableofcontents

% In a list of figures, the complete caption will be included. To prevent this,
% ALWAYS add a short description in the caption!
%
%  \caption[short description]{elaborate description}
%
% If you do, only the short description will be used in the list of figures

\listoffigures

% If you included tables and/or source code listings, uncomment the appropriate
% lines.
%\listoftables
%\listoflistings

% Als je een lijst van afkortingen of termen wil toevoegen, dan hoort die
% hier thuis. Gebruik bijvoorbeeld de ``glossaries'' package.
% https://www.overleaf.com/learn/latex/Glossaries

%---------- Kern ---------------------------------------------------------------

\mainmatter{}

% De eerste hoofdstukken van een bachelorproef zijn meestal een inleiding op
% het onderwerp, literatuurstudie en verantwoording methodologie.
% Aarzel niet om een meer beschrijvende titel aan deze hoofdstukken te geven of
% om bijvoorbeeld de inleiding en/of stand van zaken over meerdere hoofdstukken
% te verspreiden!

%%=============================================================================
%% Inleiding
%%=============================================================================

\chapter{\IfLanguageName{dutch}{Inleiding-TODO}{Introduction}}%
\label{ch:inleiding-TODO}


De inleiding moet de lezer net genoeg informatie verschaffen om het onderwerp te begrijpen en in te zien waarom de onderzoeksvraag de moeite waard is om te onderzoeken. In de inleiding ga je literatuurverwijzingen beperken, zodat de tekst vlot leesbaar blijft. Je kan de inleiding verder onderverdelen in secties als dit de tekst verduidelijkt. Zaken die aan bod kunnen komen in de inleiding~\autocite{Pollefliet2011}:

\begin{itemize}
  \item context, achtergrond
  \item afbakenen van het onderwerp
  \item verantwoording van het onderwerp, methodologie
  \item probleemstelling
  \item onderzoeksdoelstelling
  \item onderzoeksvraag
  \item \ldots
\end{itemize}

\section{\IfLanguageName{dutch}{Probleemstelling}{Problem Statement}}%
\label{sec:probleemstelling}
In dit onderzoek wordt een oplossing gezocht voor een probleem binnen het bedrijf delaware . Meer bepaald Anthony Seys, een werknemer bij delaware die op dit moment bezig is met het automatiseren van enkele processen, zou graag een mogelijke oplossing voorgesteld krijgen voor deze probleemstelling. Het probleem dat zich voordoet heeft te maken met het automatiseren van firewall regels. Concreet is er een webportal, waar klanten firewall regels kunnen meegeven. Deze regels worden vervolgens geconfigureerd en doorgegeven aan de firewall. Bij het controleren van deze gegevens vooraleer deze bij de firewall terecht komen, ondervindt men problemen. De configuratie gebeurt met een automatisatie tool genaamd Ansible. Deze tool laat niet toe om controles uit te voeren vooraleer de zaken worden geconfigureerd. Hierdoor is een oplossing nodig   die het mogelijk maakt om alle gegevens, die door de klant worden meegegeven, via de website te controleren en na te gaan of deze correct zijn. Hiervoor moet er een toolset gezocht worden die dit mogelijk maakt en die daarbijkomend ook kan samenwerken met Ansible. Dit onderzoek kan ertoe leiden dat delaware in de toekomst snel en efficiënt klanten kan verder helpen met het toevoegen van firewall regels, onafhankelijk van de soort firewall die er draait. Deze oplossing / toolset zou er tevens ook voor kunnen zorgen dat er tijdens dit proces geen tussenkomst van delaware meer zal nodig zijn, maar dat de klant dit makkelijk en volledig autonoom kan uitvoeren. Als gevolg zullen  werknemers meer tijd hebben voor het oplossen van grotere problemen. \newline



% Uit je probleemstelling moet duidelijk zijn dat je onderzoek een meerwaarde heeft voor een concrete doelgroep. De doelgroep moet goed gedefinieerd en afgelijnd zijn. Doelgroepen als ``bedrijven,'' ``KMO's'', systeembeheerders, enz.~zijn nog te vaag. Als je een lijstje kan maken van de personen/organisaties die een meerwaarde zullen vinden in deze bachelorproef (dit is eigenlijk je steekproefkader), dan is dat een indicatie dat de doelgroep goed gedefinieerd is. Dit kan een enkel bedrijf zijn of zelfs één persoon (je co-promotor/opdrachtgever).

\section{\IfLanguageName{dutch}{Onderzoeksvraag}{Research question}}%
\label{sec:onderzoeksvraag-TODO}
tijdens dit onderzoek zal volgende onderzoeksvraag worden beantwoord : "Welke toolset kan je gebruiken om een firewall regel request weg te schrijven naar een Network Virtual Appliance en de nodige controles uit te voeren?"
\section{\IfLanguageName{dutch}{Onderzoeksdoelstelling}{Research objective}}%
\label{sec:onderzoeksdoelstelling}
Het uiteindelijke doel van dit onderzoek is om een oplossing te vinden voor het bovenstaande beschreven probleem. Concreet zal er een proof-of-concept worden opgesteld. Hierin zal een simpel netwerk worden opgezet in Azure dat bestaat uit een firewall en twee virtuele machines elk in hun eigen subnet. Elk van deze apparaten zal worden geconfigureerd en geautomatiseerd met Ansible. Met als uiteindelijke doel een manier te vinden om alle gegevens , die afkomstig zijn van een website waarop de klant een firewall regel request doet , te controleren en de nodige configuraties al dan niet te laten doorvoeren. Op het moment dat er een oplossing wordt gevonden waarbij het mogelijk is om de gewenste criteria te kunnen controleren en daarna de configuratie te kunnen automatiseren, wordt dit beschouwd als een succesvolle oplossing. 

Wat is het beoogde resultaat van je bachelorproef? Wat zijn de criteria voor succes? Beschrijf die zo concreet mogelijk. Gaat het bv.\ om een proof-of-concept, een prototype, een verslag met aanbevelingen, een vergelijkende studie, enz.

\section{\IfLanguageName{dutch}{Opzet van deze bachelorproef- TODO}{Structure of this bachelor thesis}}%
\label{sec:opzet-bachelorproef-TODO}

% Het is gebruikelijk aan het einde van de inleiding een overzicht te
% geven van de opbouw van de rest van de tekst. Deze sectie bevat al een aanzet
% die je kan aanvullen/aanpassen in functie van je eigen tekst.

De rest van deze bachelorproef is als volgt opgebouwd:

In Hoofdstuk~\ref{ch:stand-van-zaken} wordt een overzicht gegeven van de stand van zaken binnen het onderzoeksdomein, op basis van een literatuurstudie.

In Hoofdstuk~\ref{ch:methodologie} wordt de methodologie toegelicht en worden de gebruikte onderzoekstechnieken besproken om een antwoord te kunnen formuleren op de onderzoeksvragen.

% TODO: Vul hier aan voor je eigen hoofstukken, één of twee zinnen per hoofdstuk

In Hoofdstuk~\ref{ch:conclusie}, tenslotte, wordt de conclusie gegeven en een antwoord geformuleerd op de onderzoeksvragen. Daarbij wordt ook een aanzet gegeven voor toekomstig onderzoek binnen dit domein.
\chapter{\IfLanguageName{dutch}{Stand van zaken}{State of the art}}%
\label{ch:stand-van-zaken}

% Tip: Begin elk hoofdstuk met een paragraaf inleiding die beschrijft hoe
% dit hoofdstuk past binnen het geheel van de bachelorproef. Geef in het
% bijzonder aan wat de link is met het vorige en volgende hoofdstuk.

% Pas na deze inleidende paragraaf komt de eerste sectiehoofding.

Dit hoofdstuk bevat je literatuurstudie. De inhoud gaat verder op de inleiding, maar zal het onderwerp van de bachelorproef *diepgaand* uitspitten. De bedoeling is dat de lezer na lezing van dit hoofdstuk helemaal op de hoogte is van de huidige stand van zaken (state-of-the-art) in het onderzoeksdomein. Iemand die niet vertrouwd is met het onderwerp, weet nu voldoende om de rest van het verhaal te kunnen volgen, zonder dat die er nog andere informatie moet over opzoeken \autocite{Pollefliet2011}.

Je verwijst bij elke bewering die je doet, vakterm die je introduceert, enz.\ naar je bronnen. In \LaTeX{} kan dat met het commando \texttt{$\backslash${textcite\{\}}} of \texttt{$\backslash${autocite\{\}}}. Als argument van het commando geef je de ``sleutel'' van een ``record'' in een bibliografische databank in het Bib\LaTeX{}-formaat (een tekstbestand). Als je expliciet naar de auteur verwijst in de zin, gebruik je \texttt{$\backslash${}textcite\{\}}.
Soms wil je de auteur niet expliciet vernoemen, dan gebruik je \texttt{$\backslash${}autocite\{\}}. In de volgende paragraaf een voorbeeld van elk.

\textcite{Knuth1998} schreef een van de standaardwerken over sorteer- en zoekalgoritmen. Experten zijn het erover eens dat cloud computing een interessante opportuniteit vormen, zowel voor gebruikers als voor dienstverleners op vlak van informatietechnologie~\autocite{Creeger2009}.
\pagebreak
\subsection{Wat is een Network Virtual Appliance in Azure?}
Voor het verdere verloop van dit onderzoek is het zeer belangrijk om te begrijpen wat een Azure Network Virtual Appliance (NVA) juist inhoud en wat het nut hiervan is. Een NVA is een virtuele machine met een user kernel , OS etc. die worden aangeboden in Azure door verschillende grote vendors. In dit onderzoek wordt er specifiek gebruik gemaakt van een Network Virtual Appliance in de vorm van een Fortigate firewall. Fortigate is een van de grotere spelers wanneer het gaat om firewalls. Specifiek wordt er voor Fortigate gekozen omdat dit de meest voorkomende firewall is bij de klanten van delaware. Microsoft biedt deze optie aan binnen Azure zodat het makkelijker zou zijn voor een klant om de overstap te maken naar de cloud. Dit omwille van het feit dat dezelfde software kan gebruilt worden als on premise. Dit zorgt ervoor dat er geen mensen zouden moeten omgeschoold worden om gebruik te maken van andere software. Op deze manier krijgt de klant alle voordelen die Azure te bieden heeft in combinatie met de vertrouwde configuratie en software. Naast Fortigate zijn er nog andere populaire keuzes zoals Cisco, Check Point, Barracuda etc. \autocite{MicrosoftNVA} 
Microsoft biedt binnen Azure ook zelf een firewall aan , genaamd Azure firewall. Een Azure firewall is een oplossing die Microsoft aanbiedt voor het beveiligen van virtuele netwerken binnen Azure. \autocite{Cooke} Het is een volledig alleenstaande firewall, die onderdeel uitmaakt van Azures Platform as a Service. \newline \autocite{Seferlis2018} Azure firewall biedt een scala aan verschillende oplossingen voor allerlei problemen, door de aanwezigheid van de vele functies. Dit valt echter buiten de scope van dit onderzoek. Aan de werking van Azure firewall kan namelijk een heel nieuw onderzoek gewijd worden. Deze optie zal niet worden bekeken voor dit onderzoek aangezien delaware meestal niet voor deze oplossing zal kiezen omwille van de hoge kostprijs. 

\subsection*{Wat is Infrastructure-as-Code?}
Infrastructure-as-Code (IaC) zal tijdens de opbouw van het proof-of-concept een belangrijke rol spelen voor de automatisatie en configuratie van de firewall. Infrastructure-as-Code is het bouwen en onderhouden van infrastructuur aan de hand van code. Dit is een technologie die ontstaan is door de opmars van de cloud. Vroeger was code enkel voor software en was er hardware nodig voor de infrastructuur. Veel zaken in de cloud zijn nu virtueel. Dat zorgt ervoor dat IaC mogelijk is. Voor het bouwen van de infrastructuren wordt er gebruik gemaakt van templates die de infrastructuur omschrijven als een object. Deze objecten worden omschreven in YML-, JSON- of XML-formaat. Zo biedt het ook enorm veel voordelen. Het interessantste voordeel voor dit onderzoek is dat, door middel van IaC, de infrastructuur zeer makkelijk uit te rollen valt naar meerdere omgevingen. \autocite{Bulthuis} Het biedt ook enorm veel mogelijkheden op vlak van uitbreiding en flexibiliteit. Het is steeds mogelijk je infrastructuur zeer snel aan te passen en uit te breiden. \autocite{Morris2016}  Voor dit onderzoek wordt er specifiek gezocht naar een oplossing in Azure met potentieel verschillende vormen firewalls. 
Hiervoor zijn verschillende oplossingen. Terraform kan worden gebruikt voor het managen en configureren van servers binnen Azure, AWS en Google Cloud. \autocite{IONOS2019} \autocite{Janashia2020} Terraform bevat wel enkele beperkingen die een groot struikelblok zijn voor de use case van dit onderzoek. --TODO--

Een andere nuttige tool voor de gewenste use case is Ansible. Dit is een open source tool die het mogelijk maakt de configuratie van virtuele machines ,zowel on premise als in de cloud , te automatiseren. \autocite{Hat} Ansible biedt een catalogus aan van verschillende 




%%=============================================================================
%% Methodologie
%%=============================================================================

\chapter{\IfLanguageName{dutch}{Methodologie}{Methodology}}%
\label{ch:methodologie}
Dit onderzoek zal in verschillende delen verlopen. In hoofdstuk twee werd er vooral informatie vergaard om het probleem zo goed mogelijk te begrijpen. Deze informatie werd verkregen aan de hand van een relevante literatuurstudie. Ook zal er beroep gedaan worden op interviews met belanghebbenden om op deze manier een requirements-analyse te kunnen uitvoeren. Dit zal ervoor zorgen dat er een optimale oplossing komt voor het beschreven probleem. De informatie wordt vervolgens gebruikt voor het opmaken van een Proof-of-Concept, waarin alle informatie zal toegepast worden in de praktijk. Dit Proof-of-Concept zal uiteindelijk ook het eindproduct zijn, waarin de oplossing van het probleem wordt weergegeven. Hierover zal ook een presentatie gegeven worden. Voor dit onderdeel wordt er beroep gedaan op communicatie met de copromoter en het bedrijf waarvoor dit onderzoek wordt uitgevoerd. Hiervoor zal een uitgebreide verzameling van software en hardware nodig zijn. Zo zal het mogelijk moeten zijn om gebruik te maken van alle Microsoft Azure features voor een zo'n specifiek mogelijke oplossing. Op deze manier kan er een realistisch scenario worden opgebouwd. Verder zal er ook eventueel beroep gedaan worden op een webapplicatie waarmee het mogelijk is firewall regels in te geven. Dit zal een zeer simpele applicatie zijn, geschreven in JavaScript. Deze applicatie is essentieel voor het oplossen van het gegeven probleem. Het is de bedoeling dat een klant aan de hand van een webapplicatie firewall regels kan doorsturen naar een firewall. Deze regels zullen worden opgenomen in een JSON-file. Op deze manier kan deze makkelijk worden geïmplementeerd in een script. Deze scripts worden gemaakt met Ansible in combinatie met een andere tool.  
Het zou kunnen dat tijdens het uitvoeren van het effectief onderzoek een betere oplossing wordt gevonden.
Concreet wordt er een Azure netwerk opgezet met een Network Virtual Appliance van het merk Fortigate. Op dit netwerk zal dan ook een webapplicatie draaien. Vervolgens worden er templates gebouwd, in verschillende Infrastructure as Code tools, voor het deployen van de gewenste firewall regels. Vervolgens zal er ook gekeken worden of deze manier van werken ook toegepast kan worden op een firewall van een andere vendor zoals Cisco. Voor dit onderdeel wordt ongeveer 40 uur geschat. Ten slotte worden de resultaten geanalyseerd en samen gebundeld in een concrete conclusie. 
%% TODO: Hoe ben je te werk gegaan? Verdeel je onderzoek in grote fasen, en
%% licht in elke fase toe welke stappen je gevolgd hebt. Verantwoord waarom je
%% op deze manier te werk gegaan bent. Je moet kunnen aantonen dat je de best
%% mogelijke manier toegepast hebt om een antwoord te vinden op de
%% onderzoeksvraag.





% Voeg hier je eigen hoofdstukken toe die de ``corpus'' van je bachelorproef
% vormen. De structuur en titels hangen af van je eigen onderzoek. Je kan bv.
% elke fase in je onderzoek in een apart hoofdstuk bespreken.

%\input{...}
%\input{...}
%...

\input{conclusie}

%---------- Bijlagen -----------------------------------------------------------

\appendix

\chapter{Code}

\section{DockerFile}
\label{code:DockerFile}
\begin{lstlisting}
    # VM version + OS
FROM ubuntu:22.04

RUN apt-get update && apt-get upgrade -y
#Installation python + usefull python tools
RUN apt-get -y install python3 python3-nacl python3-pip libffi-dev vim

# ENV ANSIBLE_HOST_KEY_CHECKING=False
#Declare Timezone
ENV TZ="Europe/Brussels"

#Install pip 
RUN python3 -m pip install --upgrade pip
#Installation of needed + usefull software
RUN pip3 install ansible jmespath paramiko pandas openpyxl ansible-pylibssh

RUN apt-get install -y sshpass
#Installation of ansible module needed to configure the firewall using ansible + module to use azure cli with ansible
RUN ansible-galaxy collection install fortinet.fortios:2.2.2 azure.azcollection
#NETCOMMON VERSION NEEDS TO BE DOWNGRADED TO 4.1.0 else everything breaks
RUN ansible-galaxy collection install ansible.netcommon:4.1.0 
RUN pip3 install -r ~/.ansible/collections/ansible_collections/azure/azcollection/requirements-azure.txt
RUN apt-get update
# Installation powershell for linux
RUN apt-get install -y wget apt-transport-https software-properties-common
RUN apt-get install wget
RUN wget -q "https://packages.microsoft.com/config/ubuntu/$(lsb_release -rs)/packages-microsoft-prod.deb"
RUN dpkg -i packages-microsoft-prod.deb
RUN apt-get update
RUN apt-get install -y powershell
# add the needed credentials to correct path. This is needed to use the azcollection.
COPY ./credentials /root/.azure/
COPY ./azureProfile.json /root/.azure/
\end{lstlisting}

\section{main.ps1}
\label{code:main.ps1}
\begin{lstlisting}[caption={main.ps1 Powershell script}]

###################################################################################
#                                                                                 #
#                                                                                 #
#                                                                                 #  
#                           SCRIPT FOR SERVICE CONTROL                            #                  
#                                                                                 #
#                                                                                 #  
#                                                                                 #
###################################################################################

#Declare name for file 
$varServ = "variables.yml"

## We want to create output file , so that we can use this file to configure the firewall using an ansible playbook
## Check if file already exists
if( -not(Test-Path -Path projects\fortigate_policy\$varServ -PathType Leaf)) 
{
    # If file does not exist then create the file 
    try {
        $null = New-Item -ItemType File -Path $varServ -Force -ErrorAction Stop
        Write-Host "The file [$varServ] has been created."
    }
    catch {
        throw $_.Exception.Message
}
}
else{
    Clear-Content $varServ
}

## fetching date from json that contains data of input user
$input = (Get-Content "main.json" -Raw) | ConvertFrom-Json

"host: " + $input.Host | Add-Content $varServ

## Run playbook to update json file with newest data.
ansible-playbook -i inventory.ini monitoringServ.yml

## fetching data from json that contains data of firewall into variable
$jsonFireServ = (Get-Content "servicesInfo.json" -Raw) | ConvertFrom-Json



## Declare array and fill it up with all the port numbers from input user
$port = @()
foreach($obj in $input.Service){
    $port += $obj.Value
}

## Declare array and fill it up with all the service names from input user
$nameServ = @()
foreach($obj in $input.Service){
    $nameServ += $obj.Description
}

## Declare array and fill it up with all the protocols from input user
$ProtoServ = @()
foreach($obj in $input.Service){
    $ProtoServ += $obj.Protocol
}

## Declaration of arrays used to seperate tcp data from udp data

### Arrays which are being filled with existing TCP services 
$tcpNameEx = @()
$tcpPortEx = @()

### Arrays which are being filled with non-existing TCP services
$tcpPortNex = @()
$tcpNameNex = @()

### Arrays which are being filled with existing UDP services
$udpNameEx = @()
$udpPortEx = @()

### Arrays which are being filled with non-existing UDP services
$udpPortNex = @()
$udpNameNex = @()

## Check if the specified service already exists yes or no. 
## Boolean used to indicate if the service already exists or not.
[Boolean]$bestaatAl = 0
foreach($elem in $input.Service){
    foreach ($obj in $jsonFireServ.meta.results)
    {
## If it already exists AND it is a TCP port add these objects to tcp exist array. 
        if($obj.{tcp-portrange} -eq $elem.Value -and $elem.Protocol -eq "TCP")
        {
            $bestaatAl = 1
            $tcpNameEx += $obj.name
            $tcpPortEx += $obj.{tcp-portrange}
        }
## If it already exists AND it is a UDP port add these object to udp exist array.
        if($obj.{udp-portrange} -eq $elem.Value -and $elem.Protocol -eq "UDP")
        {
            $bestaatAl = 1 
            $udpPortEx += $obj.{udp-portrange}
            $udpNameEx += $obj.name
            
        }
    }
## If it does not already exist AND it is a TCP port add these object to tcp non-exist array.
    if($bestaatAl -eq 0 -and $elem.Protocol -eq "TCP")
    {
      
        $tcpNameNex += $elem.Description
        $tcpPortNex += $elem.Value
    }

## If it does not already exist AND it is a UDP port add these objects to udp non-exist array.
    elseif($bestaatAl -eq 0 -and $elem.Protocol -eq "UDP")
    {
        $udpNameNex += $elem.Description
        $udpPortNex += $elem.Value
    }
## reset the boolean
    $bestaatAl = 0

}

## Formatting to get 2 seperate dictionaries into yaml file, for TCP and UDP

## New dictionary start
"RG_name: RG_Sander.VanNoten" | Add-Content $varServ
"keyvault_name: kv-weu-vans" | Add-Content $varServ
"configTCP: " | Add-Content $varServ

## Add the already existing TCP ports to the yaml file in the correct format with the correct name
for ($num = 0 ; $num -le $tcpNameEx.Length-1; $num++ )
{
"    - name: "+$tcpNameEx[$num]| Add-Content $varServ
"      port: "+$tcpPortEx[$num] | Add-Content $varServ
}

## Add the not yet existing TCP ports to the yaml file in the correct format
for ($num = 0 ; $num -le $tcpNameNex.Length-1; $num++ )
{
"    - name: TCP/"+$tcpPortNex[$num]| Add-Content $varServ
"      port: "+$tcpPortNex[$num] | Add-Content $varServ
"      description: "+$tcpNameNex[$num] | Add-Content $varServ
}

## add new line
"`n" | Add-Content $varServ
## start new dictionary
"configUDP: " | Add-Content $varServ
## Add the already existing UDP ports to the yaml file in the correct format
for ($num = 0 ; $num -le $udpNameEx.Length-1; $num++ )
{
"    - name: "+$udpNameEx[$num]| Add-Content $varServ
"      port: "+$udpPortEx[$num] | Add-Content $varServ
}

## Add the not yet existing UDP ports to the yaml file in the correct format
for ($num = 0 ; $num -le $udpNameNex.Length-1; $num++ )
{
"    - name: UDP/"+$udpPortNex[$num]| Add-Content $varServ
"      port: "+$udpPortNex[$num] | Add-Content $varServ
"      description: "+$udpNameNex[$num] | Add-Content $varServ
}
## If everything is done , add the new services to the firewall. 



$quantity = @()
foreach($elem in $port)
{
    if($elem -like '*-*')
    {
        $quantity += $elem.Split('-')
    }
}

for ($num = 1 ; $num -le $quantity.Length-1; $num++ )
{
        $amountOfPorts = $quantity[$num] - $quantity[$num-1]

        if($amountOfPorts -le 100)
        {
            Write-Host "Opening up ${amountOfPorts} ports"
        }
        else{
            Write-Error "You can only add 100 ports at a time. You are now trying to add ${amountOfPorts} ports"
            exit 10
        }
        # Dit is nodig, want volgende iteratie moet 2 sprongen maken
        $num+=1
        
    }

###################################################################################
#                                                                                 #
#                                                                                 #
#                                                                                 #  
#                           SCRIPT FOR IP CONTROL                                 #             
#                                                                                 #
#                                                                                 #  
#                                                                                 #
###################################################################################


## regex to check if an IP is of IPV4 format + check if it contains /32 prefix
$Check_32prefix = '(\d{1,3}\.){3}(\d{1,3}\/){1}32$'
## regex to check if an IP is of IPV4 format
$Check_if_ip = '(\d{1,3}\.){3}(\d{1,3})$' 

$varFile = "variables.yml"

## Run playbook to update json file with newest data.
#ansible-playbook -i inventory.ini monitoring.yml

## fetching data from json that contains data of firewall into variable
$jsonFire = (Get-Content "Addresses.json" -Raw) | ConvertFrom-Json

## fetching date from json that contains data of input user


#Declare array and fill it up with all the ip addresses that are in the firewall. 
$ipFire = @()
foreach($obj in $jsonFire.meta.results){
    $ipFire += $obj.subnet
}

## Declare array and fill it up with all the ip addresses from input user
$ip = @()
foreach($obj in $input.Address){
    $ip += $obj.Value
}

## Declare array and fill it up with all the names given to the ip addresses from input user
$jsName = @()
foreach($obj in $input.Address){
   $jsName += $obj.Name
}

## Split up the array so that it only contains IP addresses without /32 to perform check 
# $ipControl = $ip | ForEach-Object {
#         ([string]$_).Split("/")[0] 
#     } 

## Split up the array so that it only contains IP addresses without the full subnetmask to perform check
# $ipFireControl = $ipFire | ForEach-Object {
#     ([string]$_).Split(" ")[0] 
# } 

$dictionary = New-Object System.Collections.Generic.Dictionary"[String,String]" 
foreach($obj in $input.Address)
{
    $dictionary.Add($obj.Name, $obj.Value+" 255.255.255.255" )
}

$endName = @()
$endAdd = @()

$test = @()
foreach($obj in $ip)
{
    if($obj -match $Check_32prefix)
    {
        $test+= $obj.replace('/32', ' 255.255.255.255')
    }
    else{
        $test+= $obj + " 255.255.255.255"
    }
    
}

[Boolean]$bestaatAl = 0
foreach($key in $dictionary.Keys){
    foreach($obj in $jsonFire.meta.results)
    {
        if($obj.subnet -eq $($dictionary[$key]))
        {
            $bestaatAl = 1
            $endName += $obj.name
            $endAdd += $obj.subnet
        }
    }
    
    if($bestaatAl -eq 0)
    {
        $endName += $key
        $endAdd += $($dictionary[$key])

    }
    $bestaatAl = 0
}
"`n" | Add-Content $varFile
"ConfigAdd: " | Add-Content $varFile
for ($num = 0 ; $num -le $endAdd.Length-1 ; $num++ )
{
        if( $ip[$num] -match $Check_if_ip ) 
        {
            "    - name: "+$endName[$num] | Add-Content $varFile
            "      adres: "+$endAdd[$num].replace(' 255.255.255.255', '/32') | Add-Content $varFile 
        }else {
            Write-Error "$($ip[$num]) is not a valid IPaddress"
            Exit 10
         }
    }
    # If it does not include /32 but a /24 for example throw this error and exit script 
   
Write-host "Initiating playbook..."
# After all the checks are complete and successful , run the addresses playbook to add the IP addresses to the firewall

###################################################################################
#                                                                                 #
#                                                                                 #
#                                                                                 #  
#                           CREATE POLICY                                         #             
#                                                                                 #
#                                                                                 #  
#                                                                                 #
###################################################################################

"`n" | Add-Content $varFile
"ConfigPolicy: " | Add-Content $varFile
foreach($obj in $input.Policy)
{
    "    - name: "+$obj.name | Add-Content $varFile
    "      sourceInt: "+'"{{ '+$obj.sourceInt+'}}"' | Add-Content $varFile
    "      destInt: "+'"{{ '+$obj.destInt+'}}"' | Add-Content $varFile
    "      service: "+$obj.service | Add-Content $varFile
}
    "      sourceAdd: " | Add-Content $varFile
foreach($obj in $input.Policy.sourceAdd)
{

"          - name: "+$obj | Add-Content $varFile
}

"      destAdd: " | Add-Content $varFile
foreach($obj in $input.Policy.destAdd)
{

"          - name: "+$obj | Add-Content $varFile
}


foreach($obj in $jsonFire.meta.results)
{
    if ($obj.name -like $input.Policy.destAdd)
    {
        "      dstAdd: "+$obj.subnet.split(" ")[0] | Add-Content $varFile
    }
}
ansible-playbook -i inventory.ini main.yml

    
\end{lstlisting}

\section{variables.yml}
\label{code:Variables.yml}
\begin{lstlisting}[caption={Variables.yml, dictionaries met elk een eigen doel}]
host: fortigate1
RG_name: RG_Sander.VanNoten
keyvault_name: kv-weu-vans
configTCP: 
    - name: TCP/8080
      port: 8080
    - name: TCP/8080
      port: 8080
      description: bachelorproef


configUDP: 


ConfigAdd: 
    - name: dmz-web
      adres: 172.22.2.5/32
    - name: lan-host1
      adres: 172.22.1.5/32


configPolicy: 
    - name: HostToWeb
      sourceInt: "{{ internal}}"
      destInt: "{{ web}}"
      service: TCP/8080
      sourceAdd: 
          - name: host-bachelorproef
      destAdd: 
          - name: web-bachelorproef

\end{lstlisting}

\section{monitoring.yml}
\label{code:monitoring.yml}

\begin{lstlisting}[caption={monitoring.yml, staat in voor ophalen info uit keyvault}]
        - name: Get Fortigate password from keyvault
      import_tasks: /projects/fortigate_policy/keyvault.yml
      
    - name: Get Services
      fortios_configuration_fact:
        access_token: "{{ access_token }}"
        vdom: "{{ vdom }}"
        selectors: 
          - selector: firewall.service_custom
      register: resultServ
      

    - name: Create raw JSON file of the Address output
      copy:
        content: |
          {{ resultServ | to_nice_yaml }} 
        dest: /projects/fortigate_policy/servicesInfo.yaml
    - name: Create raw JSON file of the Address output
      copy:
        content: |
          {{ resultServ | to_nice_json }} 
        dest: servicesInfo.json
    - name: Get Addresses
      fortios_configuration_fact:
        access_token: "{{ access_token }}"
        vdom: "{{ vdom }}"
        selectors:
          - selector: firewall_address
      register: resultAd
      

    - name: Create raw JSON file of the Address output
      copy:
        content: |
          {{ resultAd | to_nice_json }} 
        dest: /projects/fortigate_policy/Addresses.json

\end{lstlisting}

\section{main.yml}
\label{code:main.yml}

\begin{lstlisting}[caption=main.yml]
    ---
- hosts: "{{ host }}"
  connection: httpapi 
  gather_facts: False                                        
  collections:
   - fortinet.fortios
   - azure.azcollection
  vars:
   vdom: "root"
   ansible_httpapi_use_ssl: yes
   ansible_httpapi_validate_certs: no
   ansible_httpapi_port: 8443
  vars_files: ./variables.yml
  tasks:
  - name: Get Fortigate password from keyvault
    import_tasks: /projects/fortigate_policy/keyvault.yml
     delegate_to: localhost
   
  - name: Add address to firewall 
    import_tasks: /projects/fortigate_policy/addresses.yml

  - name: Add services to firewall 
    import_tasks: /projects/fortigate_policy/services.yml
  
  - name: Add policy to firewall
    import_tasks: /projects/fortigate_policy/policy.yml
\end{lstlisting}

\section{keyvault.yml}
\label{code:keyvault.yml}
\begin{lstlisting}[caption={keyvault.yml}]
   - name: Get Key Vault by name
     azure_rm_keyvault_info:
      resource_group: "{{ RG_name }}"
      name: "{{ keyvault_name }}"
     register: keyvault
     become_user: root
   - name: Set key vault URI fact
     set_fact: keyvaulturi="{{ keyvault['keyvaults'][0]['vault_uri'] }}"

   - name: Get secret value for the ansible api key
     azure_rm_keyvaultsecret_info:
      vault_uri: "{{ keyvaulturi }}"
      name: "fgt-accesstoken"
     register: kvSecret
   
   - name: set secret fact
     set_fact: access_token="{{ kvSecret['secrets'][0]['secret'] }}"
\end{lstlisting}

\section{addresses.yml}
\label{code:addresses.yml}

\begin{lstlisting}[caption={addresses.yml}]
  - name: "Adding address to firewall"
    fortios_firewall_address:
      vdom: "{{ vdom }}" 
      state: "present"
      enable_log: true
      access_token: "{{ access_token }}"
      firewall_address:
        name: "{{ item.name }}"
        type: "ipmask"
        subnet: "{{ item.adres }}" 
    with_items: "{{ ConfigAdd }}"

\end{lstlisting}

\section{services.yml}
\label{code:services.yml}
\begin{lstlisting}[caption={services.yml}]
  - name: Configure tcp services from customer.
    fortios_firewall_service_custom:
      vdom:  "{{ vdom }}"
      state: "present"
      access_token: "{{ access_token }}"
      firewall_service_custom:
        name: "{{ item.name }}"
        category: "Network Services"
        protocol: "TCP/UDP/SCTP"
        tcp_portrange: "{{ item.port }}" 
        # default omit --> Make parameter optional
        comment: "{{ item.description | default(omit) }}"
    when: item.name != None and item.port != None
    with_items: "{{ configTCP }}"
  
  - name: Configure udp services from customer.
    fortios_firewall_service_custom:
      vdom:  "{{ vdom }}"
      state: "present"
      access_token: "{{ access_token }}"
      firewall_service_custom:
        name: "{{ item.name | default(omit)}}"
        category: "Network Services"
        protocol: "TCP/UDP/SCTP"
        udp_portrange: "{{ item.port | default(omit) }}" 
        #Default omit --> Make parameter optional
        comment: "{{ item.description | default(omit) }}"
    when: item.name != None and item.port != None
    with_items: "{{ configUDP }}"

\end{lstlisting}

\section{policy.yml}
\label{code:policy.yml}

\begin{lstlisting}[caption={policy.yml}]
   - name: Configure policy LAN to DMZ (policy 4).
     fortios_firewall_policy:
      vdom:  "{{ vdom }}"
      state: "present"
      access_token: "{{ access_token }}"
      enable_log: true
      firewall_policy:
        policyid: 0
        name: "{{ item.name }}" 
        srcintf: 
          - name: "{{ item.sourceInt }}" 
        dstintf:
          - name: "{{ item.destInt }}"
        srcaddr: "{{ item.sourceAdd }}" 
        dstaddr: "{{ item.destAdd }}"
        schedule: "always"
        service:
          - name: "{{ item.service }}"
        action: "accept"
     with_items: "{{ ConfigPolicy }}"
\end{lstlisting}

\section{main.json}
\label{code:main.json}

\begin{lstlisting}
    {
    "Policy": [
            
        { 
            "name": "HostToWeb",
            "sourceInt": "internal", 
            "destInt": "web",
            "sourceAdd":["host-bachelorproef"],
            "destAdd": ["web-bachelorproef"],
            "service": "TCP/8080"
            
        }
    ],

    "Service": [
        {
            "Cat": "Service",
            "Value": "8080",
            "Protocol": "TCP",
            "Description": "bachelorproef"

        }
    ],

    "Address": [
        {
            "Cat": "Address",
            "Value": "172.22.2.5",
            "Name": "web-bachelorproef"
            
        },
        {
            "Cat": "Address",
            "Value": "172.22.1.5",
            "Name": "host-bachelorproef"
        }
     
    ],

    "Host" : "fortigate1"
 
}
\end{lstlisting}

\chapter{Onderzoeksvoorstel}

Het onderwerp van deze bachelorproef is gebaseerd op een onderzoeksvoorstel dat vooraf werd beoordeeld door de promotor. Dat voorstel is opgenomen in deze bijlage.

%% TODO: 
\section*{Samenvatting}

In deze bachelorproef Toegepaste Informatica wordt er onderzocht welke toolset er kan gebruikt worden voor het deployen van firewall regels, afkomstig van een web request, naar een Azure firewall en/of een Network Virtual Appliance. Dit gebeurt in opdracht van het IT-bedrijf delaware. Hiervoor zijn verschillende Infrastructure-as-Code oplossingen mogelijk zoals Terraform en Bicep. Aan de hand van een proof of concept worden er templates opgemaakt en getest in de praktijk. Dit gebeurt in een Azure Netwerk met een Azure firewall en een Network Virtual Appliance. Deze proof of concept zal aantonen welke tools er uiteindelijk in de toolset zullen zitten om het probleem op te lossen en dit zo efficiënt mogelijk te laten gebeuren. 

% Verwijzing naar het bestand met de inhoud van het onderzoeksvoorstel
%---------- Inleiding ---------------------------------------------------------

\section{Introductie}%
\label{sec:introductie}



Het automatiseren van IT-processen wordt \newline steeds belangrijker voor de verdere ontwikkeling en efficiëntie van bedrijfsprocessen. Dit zorgt ervoor dat het IT-departement zich niet eerst steeds moet focussen op kleine simpele taken, zoals bijvoorbeeld het installeren en configureren van een databank. Met behulp van verschillende tools kan dit zeer snel geautomatiseerd worden. 

Dit onderzoek gaat over het deployen van firewallrules via een webapplicatie en het automatiseren hiervan. Vaak gebeurt het implementeren van deze regels nog manueel. Het komt echter steeds voor dat een klant nog bijkomende zaken nodig heeft, waarvoor een nieuwe firewallrule nodig is. Concreet wordt er dus onderzocht welke toolset geschikt is voor het deployen van deze regels, afkomstig van een web request, naar een Azure firewall en/of Network Virtual Appliance. Dit onderzoek wordt uitgevoerd in opdracht van het bedrijf delaware die graag een oplossing ziet voor dit probleem. 

Het uiteindelijke doel van dit onderzoek is het opstellen van een proof-of-concept waarin duidelijk wordt welke toolset je het beste kunt gebruiken om het probleem op te lossen, indien dit mogelijk is.


%---------- Stand van zaken ---------------------------------------------------

\section{State-of-the-art}%
\label{sec:state-of-the-art}



Dit onderzoek zal vooral gaan over het automatisatieaspect van het probleem. Meer bepaald zal er gekeken worden binnen het domein van Infrastructure as Code (IaC).
Over dit onderwerp specifiek is nog geen onderzoek terug te vinden. Er wordt gekeken naar verschillende elementen die van toepassing zijn. Wanneer er dieper ingegaan wordt op de onderzoeksvraag roept dit meteen enkele vragen op. 

\subsection*{Wat is het verschil tussen een Azure firewall en een Network Virtual Appliance?}
Voor het verdere verloop van dit onderzoek is het enkel noodzakelijk om de twee verschillende firewalls even high level te bekijken. 
Het verschil tussen beide vormen is al meermaals onderzocht. Hoewel heel wat zaken niet relevant zijn en buiten de scope van dit onderzoek vallen, komen er, voor dit onderzoek, belangrijke verschillen naar boven.

Een Azure firewall is een oplossing die Microsoft aanbiedt voor het beveiligen van virtuele netwerken binnen Azure. \autocite{Cooke} Het is een volledig alleenstaande firewall, die onderdeel uitmaakt van Azures Platform as a Service. \newline \autocite{Seferlis2018} Azure firewall biedt een scala aan verschillende oplossingen voor allerlei problemen, door de aanwezigheid van de vele functies. Dit valt echter buiten de scope van dit onderzoek. Aan de werking van Azure firewall kan namelijk een heel nieuw onderzoek gewijd worden.

Een Network Virtual Appliance (NVA) is een virtuele machine met een user kernel etc. Deze VM wordt in het geval van dit onderzoek softwarematig ingesteld als een firewall. \autocite{Awati} Op deze manier kan er gebruik gemaakt worden van firewall software van allerlei vendors. Het is ook mogelijk om deze in een Virtual Azure netwerk te plaatsen. Dit wordt meestal gebruikt wanneer iemand specifieke kennis heeft over de firewall van één specifieke vendor, maar toch in de Azure omgeving wilt werken. Dit kan handig zijn wanneer er een overschakeling wordt gemaakt van on premise naar de cloud. 
\autocite{Aviatrix} Er kan dus geconcludeerd worden dat het grootste verschil tussen beiden is dat Azure firewall een native hardwarematige firewall is binnen de Azure PaaS en dat een NVA een normale VM is die softwarematig wordt omgevormd naar een firewall. Voor dit onderzoek zal het dus van belang zijn om goed te onderscheiden naar welke firewall er rules zullen gedeployed worden. In het geval van een NVA zal het ook belangrijk zijn om te weten welke software er gedraaid wordt en van welke vendor deze afkomstig is. 

\subsection*{Wat is Infrastructure-as-Code?}
Infrastructure-as-Code is het bouwen en onderhouden van infrastructuur aan de hand van code. Dit is een technologie die ontstaan is door de opmars van de cloud. Vroeger was code enkel voor software en was er hardware nodig voor de infrastructuur. Veel zaken in de cloud zijn nu virtueel. Dat zorgt ervoor dat IaC mogelijk is. Voor het bouwen van de infrastructuren wordt er gebruik gemaakt van templates die de infrastructuur omschrijven als een object. Deze objecten worden omschreven in YML-, JSON- of XML-formaat. Zo biedt het ook enorm veel voordelen. Het interessantste voordeel voor dit onderzoek is dat, door middel van IaC, de infrastructuur zeer makkelijk uit te rollen valt naar meerdere omgevingen. \autocite{Bulthuis} Het biedt ook enorm veel mogelijkheden op vlak van uitbreiding en flexibiliteit. Het is steeds mogelijk je infrastructuur zeer snel aan te passen en uit te breiden. \autocite{Morris2016}  Voor dit onderzoek wordt er specifiek gezocht naar een oplossing in Azure met potentieel verschillende vormen firewalls. 
Hiervoor zijn verschillende oplossingen. Terraform kan worden gebruikt voor het managen en configureren van servers binnen Azure, AWS en Google Cloud. \autocite{IONOS2019} \autocite{Janashia2020} 
Bicep biedt ook dezelfde functies aan, maar is enkel te gebruiken voor Azure toepassingen. Deze twee voorbeelden zijn dan ook de meest geschikte tools voor dit onderzoek. \newline \autocite{Zerger} 


\subsection*{Hoe een firewall regel deployen naar Azure firewall?}

Voor het deployen van firewall rules naar een Azure firewall zijn er verschillende manieren om dit aan te pakken. 
Volgens Vincent Misson \autocite{Misson2021} is het een mogelijke oplossing om dit te doen met Terraform. Dit is een Infrastructure as Code tool die instaat voor het deployen en configureren van servers, verdeeld over verschillende Cloud providers en wordt geschreven in Go formaat. Vincent toont enkele voorbeelden die nuttig kunnen zijn voor het vervolg van dit onderzoek. Terraform kan je gebruiken voor deployment naar Azure, AWS en Google cloud. \autocite{Harrington2022} 
Volgens de documentatie van Microsoft \autocite{2022a} is dit ook perfect mogelijk door gebruik te maken van bicep. Ook Bicep is een Infrastructure as Code tool. Het werkt zeer gelijkaardig aan Terraform, alleen is bicep enkel bedoeld voor het deployen naar Azure. \autocite{Dave2021} 
In een andere documentatie van Microsoft \autocite{2022} wordt ook een oplossing geboden aan de hand van de Azure portal. Dit kan voor andere doeleinde interessant zijn, maar deze manier van werken zal niet aan bod komen in dit onderzoek, aangezien het op deze manier niet mogelijk is om zaken snel te automatiseren. 


\subsection*{Hoe een firewall rule deployen naar een Native Virtual Appliance}
Wanneer er gekeken wordt naar het deployen van firewall regels naar een NVA, blijkt dat dit heel vendor afhankelijk is. Er is niet meteen een centrale manier om dit te doen. Dit gebeurt steeds met de API van de vendor. Het is wel mogelijk om te achterhalen welke API er gebruikt wordt voor het configureren van de NVA in kwestie. Dit door gebruik te maken van een PowerShell commando. \autocite{Microsoft2022} Ook is het mogelijk om firewall regels te deployen tijdens het aanmaken van een NVA in de Azure portal. \autocite{Alto2022} Dit zou kunnen zorgen voor een mogelijkheid om, door middel van Bicep, de Azure portal aan te spreken en op deze manier de firewall regel te kunnen deployen. 


 

% Voor literatuurverwijzingen zijn er twee belangrijke commando's:
% \autocite{KEY} => (Auteur, jaartal) Gebruik dit als de naam van de auteur
%   geen onderdeel is van de zin.
% \textcite{KEY} => Auteur (jaartal)  Gebruik dit als de auteursnaam wel een
%   functie heeft in de zin (bv. ``Uit onderzoek door Doll & Hill (1954) bleek
%   ...'')


%---------- Methodologie ------------------------------------------------------
\section{Methodologie}%
\label{sec:methodologie}

Dit onderzoek zal in verschillende delen verlopen. In het eerste deel zal er vooral informatie worden vergaard om het probleem zo goed mogelijk te begrijpen. Deze informatie wordt verkregen aan de hand van relevante literatuurstudie. Ook zal er beroep gedaan worden op interviews met belanghebbenden om op deze manier een requirements-analyse te kunnen uitvoeren. Dit zal ervoor zorgen dat er een optimale oplossing komt voor het beschreven probleem. Dit onderdeel van het onderzoek zal vermoedelijk 20 uur in beslag nemen. 
De informatie wordt vervolgens gebruikt voor het opmaken van een Proof-of-Concept, waarin alle informatie zal toegepast worden in de praktijk. Dit Proof-of-Concept zal uiteindelijk ook het eindproduct zijn, waarin de oplossing van het probleem wordt weergegeven. Hierover zal ook een presentatie gegeven worden. Voor dit onderdeel wordt er beroep gedaan op communicatie met de copromoter en het bedrijf waarvoor dit onderzoek wordt uitgevoerd. Hiervoor zal een uitgebreide verzameling van software en hardware nodig zijn. Zo zal het mogelijk moeten zijn om gebruik te maken van alle Microsoft Azure features voor een zo'n specifiek mogelijke oplossing. Op deze manier kan er een realistisch scenario worden opgebouwd. Verder zal er ook eventueel beroep gedaan worden op een webapplicatie waarmee het mogelijk is firewall regels in te geven. Dit zal een zeer simpele applicatie zijn, geschreven in JavaScript. Deze applicatie is essentieel voor het oplossen van het gegeven probleem. Het is de bedoeling dat een klant aan de hand van een webapplicatie firewall regels kan doorsturen naar een firewall. Deze regels zullen worden opgenomen in een JSON-file. Op deze manier kan deze makkelijk worden geïmplementeerd in een script. Deze scripts zullen hoogstwaarschijnlijk worden gemaakt in Bicep of Terraform.  
Het zou kunnen dat tijdens het uitvoeren van het effectief onderzoek een betere oplossing wordt gevonden.
Concreet wordt er een Azure netwerk opgezet met een Azure firewall en een Azure netwerk opgezet die gebruik maakt van een Network Virtual Appliance. Op beide netwerken zal dan ook een webapplicatie draaien. Vervolgens worden er templates gebouwd, in verschillende Infrastructure as Code tools, voor het deployen van de gewenste firewall regels. Vervolgens zal er ook gekeken worden of deze manier van werken ook toegepast kan worden op een Network Security Group (NSG). Voor dit onderdeel wordt ongeveer 40 uur geschat. Ten slotte worden de resultaten geanalyseerd en samen gebundeld in een concrete conclusie. 

%---------- Verwachte resultaten ----------------------------------------------
\section{Verwacht resultaat, conclusie}%
\label{sec:verwachte_resultaten}

Tijdens het uitvoeren van dit onderzoek wordt er verwacht dat er een oplossing wordt gevonden voor het omschreven probleem. Vermoedelijk zullen hier drie of vier verschillende soorten tools voor nodig zijn. Dit zal er bijgevolg voor zorgen dat het niet mogelijk zal zijn om in alle gevallen het proces volledig te automatiseren. De automatisatie naar een Azure firewall zou moeten lukken door middel van tools, zoals Terraform en Bicep. In tegenstelling tot het deployen van firewall regels naar een NVA, zal dit niet vanzelfsprekend zijn. Dit komt door de verschillende factoren die dit resultaat kunnen beïnvloeden.  Voor delaware zal dit onderzoek ertoe leiden dat ze in het verdere toekomst snel en efficiënt de klant kunnen verder helpen met het toevoegen van firewall regels, onafhankelijk van welke soort firewall er draait. Dit zou ervoor kunnen zorgen dat er tijdens dit proces geen tussenkomst van delaware meer zal nodig zijn, maar dat de klant dit makkelijk en volledig autonoom kan uitvoeren. Dit zorgt ervoor dat de werknemers meer tijd zullen hebben voor het oplossen van grotere problemen. 

%%---------- Andere bijlagen --------------------------------------------------
% TODO: Voeg hier eventuele andere bijlagen toe. Bv. als je deze BP voor de
% tweede keer indient, een overzicht van de verbeteringen t.o.v. het origineel.
%\input{...}

%%---------- Backmatter, referentielijst ---------------------------------------

\backmatter{}

\setlength\bibitemsep{2pt} %% Add Some space between the bibliograpy entries
\printbibliography[heading=bibintoc]

\end{document}
