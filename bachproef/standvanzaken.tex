\chapter{\IfLanguageName{dutch}{Stand van zaken}{State of the art}}%
\label{ch:stand-van-zaken}

In dit hoofdstuk wordt er gekeken naar alle informatie die er nodig is voor het uitvoeren van het onderzoek. Dit zal gebeuren aan de hand van een relevante literatuurstudie in combinatie met een kort interview met de belanghebbende Zodat er een korte requirement analyse kan worden opgesteld. Dit kan dan gebruikt worden als eerste stap voor het opstellen van de proof-of-concept. \newline

De onderdelen die besproken zullen worden zijn belangrijk voor het verdere verloop van het onderzoek. Als eerst wordt er gekeken naar wat een Network Virtual Appliance juist inhoudt en wat het nut hiervan is. Vervolgens wordt er wat dieper ingegaan op het begrip Infrastructure As Code en welke tools hiertoe behoren. Verder zal er al kort worden beschreven welke controles er nodig zijn en hoe deze kunnen worden toegepast. Na een korte omschrijving van welke tools er beschikbaar zijn voor het oplossen van het probleem , wordt er specifiek gekeken naar wat deze tools juist zijn en hoe ze zullen worden gebruikt. Vervolgens wordt er gekeken naar hoe deze tools het best kunnen samenwerken en welke problemen dit met zich meebrengt. Tot slot wordt er gekeken naar de virtualisatie van machines die nodig zullen zijn en welke vorm het best geschikt is voor het onderzoek.
% Tip: Begin elk hoofdstuk met een paragraaf inleiding die beschrijft hoe
% dit hoofdstuk past binnen het geheel van de bachelorproef. Geef in het
% bijzonder aan wat de link is met het vorige en volgende hoofdstuk.

% Pas na deze inleidende paragraaf komt de eerste sectiehoofding.

% Dit hoofdstuk bevat je literatuurstudie. De inhoud gaat verder op de inleiding, maar zal het onderwerp van de bachelorproef *diepgaand* uitspitten. De bedoeling is dat de lezer na lezing van dit hoofdstuk helemaal op de hoogte is van de huidige stand van zaken (state-of-the-art) in het onderzoeksdomein. Iemand die niet vertrouwd is met het onderwerp, weet nu voldoende om de rest van het verhaal te kunnen volgen, zonder dat die er nog andere informatie moet over opzoeken \autocite{Pollefliet2011}.

% Je verwijst bij elke bewering die je doet, vakterm die je introduceert, enz.\ naar je bronnen. In \LaTeX{} kan dat met het commando \texttt{$\backslash${textcite\{\}}} of \texttt{$\backslash${autocite\{\}}}. Als argument van het commando geef je de ``sleutel'' van een ``record'' in een bibliografische databank in het Bib\LaTeX{}-formaat (een tekstbestand). Als je expliciet naar de auteur verwijst in de zin, gebruik je \texttt{$\backslash${}textcite\{\}}.
% Soms wil je de auteur niet expliciet vernoemen, dan gebruik je \texttt{$\backslash${}autocite\{\}}. In de volgende paragraaf een voorbeeld van elk.

% \textcite{Knuth1998} schreef een van de standaardwerken over sorteer- en zoekalgoritmen. Experten zijn het erover eens dat cloud computing een interessante opportuniteit vormen, zowel voor gebruikers als voor dienstverleners op vlak van informatietechnologie~\autocite{Creeger2009}.
% \pagebreak
\newpage
\section{Wat is een Network Virtual Appliance in Azure?}
Om dit onderzoek te starten is het belangrijk om te begrijpen wat een Azure Network Virtual Appliance (NVA) juist inhoudt en wat het nut hiervan is.

Microsoft biedt binnen Azure zelf een Firewall oplossing aan, namelijk de Azure Firewall. Dit maakt het mogelijk om virtuele netwerken binnen Azure te beveiligen. \autocite{Cooke} Dit is een volledig alleenstaande firewall die onderdeel uitmaakt van Azures Platform as a Service. Azure firewall biedt een scala aan van verschillende oplossingen voor allerlei problemen, door de aanwezigheid van de vele functies. \autocite{Seferlis2018} Hoewel Azure firewall een goede keuze zou zijn, valt dit toch buiten de scope van het onderzoek. Niet enkel omdat dit redelijk breed is en een eigen onderzoek kan hebben, maar ook omdat delaware zelf niet zal kiezen voor deze optie omwille van de hoge kostprijs.

Hierdoor zullen bedrijven, en dus ook delaware, sneller opteren voor een NVA. Een NVA is een virtuele machine met een user kernel, OS etc. die kan gebruikt worden binnen Azure. NVA's worden aangeboden binnen Azure door Microsoft zodat het voor de klant makkelijker en goedkoper is om de overstap te maken van on premise naar de cloud. Dit omwille van het feit dat dezelfde software wordt gebruikt zowel on premise als in de cloud. Hierdoor moet men geen opleiding voorzien om gebruik te maken van een andere software. Dit biedt de klant alle voordelendie Azure te bieden heeft samen met de vertrouwde configuratie en software. Eén van de grootste spelers op vlak van NVA's is Fortinet. Echter zijn er ook andere spelers, namelijk Cisco, Check Point, Barracuda etc. \autocite{MicrosoftNVA} Kijkende naar delaware, is de Fortigate firewall de meest voorkomende firewall en zal deze ook gebruikt worden binnen dit onderzoek. 

\section{Wat is Infrastructure-as-Code?}
Infrastructure-as-Code (IaC) zal tijdens de opbouw van de proof-of-concept een belangrijke rol spelen voor de automatisatie en configuratie van de firewall. Infrastructure-as-Code is het bouwen en onderhouden van de infrastructuur aan de hand van code. Dit is een technologie die ontstaan is door de opmars van de cloud. Vroeger werd code enkel gebruikt voor het maken van software en was er nood aan hardware voor de infrastructuur. Vandaag de dag worden databanken en servers etc. steeds vaker in de cloud geplaatst. Dat zorgt ervoor dat er steeds meer vraag is naar IaC. Voor het bouwen van de infrastructuren wordt er gebruik gemaakt van templates die de infrastructuur omschrijven als een object. Deze objecten worden omschreven in YML-, JSON- of XML-formaat. Dit zorgt voor vele voordelen. Het interessantste voordeel voor dit onderzoek is dat, door middel van IaC, de infrastructuur makkelijk uit te rollen valt naar meerdere omgevingen \autocite{Bulthuis}. Het biedt ook enorm veel mogelijkheden op vlak van uitbreiding en flexibiliteit. Het is steeds mogelijk de infrastructuur aan te passen en uit te breiden \autocite{Morris2016}. Daarnaast heeft IaC wel het nadeel dat het een grote "learning curve" bevat. Vooraleer er zaken kunnen uitgerold en aangepast worden in code moeten de templates geschreven worden en dit kan een grote hoeveelheid tijd in beslag nemen. Soms is het makkelijker handmatig een aanpassing door te voeren op de machine. Bijgevolg kan dit zorgen voor conflicten met de IaC waardoor er zaken worden overschreven. Voor dit onderzoek wordt er specifiek gezocht naar een oplossing in Azure die de mogelijkheid biedt verschillende firewalls van meerdere vendors te configureren. 
Hiervoor zijn verschillende oplossingen. Een van de deze mogelijkheden is Terraform, die instaat voor het managen en configureren van servers binnen Azure, AWS en Google Cloud \autocite{IONOS2019} \autocite{Janashia2020}. Echter bevat Terraform enkele beperkingen die gezien kunnen worden als een struikelblok voor dit onderzoek. Terraform is namelijk een zeer goede oplossing voor use cases die volledig in IaC worden geïmplementeerd. Echter zal er binnen de use case van dit onderzoek onderdelen nog steeds manueel geconfigureerd worden op de firewall, waardoor niet alles via IaC zal verlopen. Waarom is dit dus niet mogelijk in Terraform? Terraform werkt met een state file. Deze state file houdt alle configuraties die gemaakt wordem in IaC bij. Wanneer iemand vervolgens manueel een aanpassing maakt op de firewall , Zal deze manuele aanpassing overschreven worden van het moment dat de Terraform code opnieuw wordt uitgevoerd. Dit vormt dus een struikelblok, die op te lossen valt door gebruik te maken van Ansible.  \autocite{Anthony}

\subsubsection{Wat is Ansible?}
\label{sec:Wat is ansible}
Ansible is een open source tool die het mogelijk maakt de configuratie van virtuele machines ,zowel on premise als in de cloud , te automatiseren \autocite{Hat}. Ansible biedt een catalogus aan van verschillende modules die kunnen gebruikt worden voor allerlei doeleinden. Zo is er een module voor het configureren van Fortigate firewalls. Deze module zal in rekening worden genomen bij het maken van de proof-of-concept \autocite{Fortinet2020}. Ansible kan een oplossing bieden op het probleem dat zich voordoet binnen Terraform. Ansible zal enkel de gewenste configuratie doorvoeren, ongeacht of deze al bestaan ja of nee. Wanneer deze al bestaat wordt deze dus niet overschreven. Wat wel het geval zal zijn bij het gebruik van Terraform. Ansible lijkt bijgevolg de meest vatbare oplossing voor het configureren van de firewall. Ansible maakt gebruik van YAML files voor de configuratie van de infrastructuur. Deze YAML files hebben een zeer stricte syntax en moeten dus steeds juist geformateerd worden. Wanneer dit niet het geval is zal Ansible dit niet kunnen interpreteren en zal er een error getoond worden.

\subsection{Controles?}

Zoals reeds besproken in de probleemstelling , is het de bedoeling dat een klant aan de hand van een webportaal firewall regels automatisch kan doorgeven aan de firewall. Het probleem waar delaware tegenaan loopt binnen deze use case is dat men wenst controle uit te voeren op de data die wordt meegegeven door de klant. Eerder werd Ansible reeds als oplossing naar boven gekomen voor het configureren van de firewall. Echter biedt Ansible geen mogelijkheid aan om controle uit te voeren op de input meegeven door de klant. Er is dus nood aan een alternatieve extra stap. De input van de klant zal geformateerd worden naar JSON formaat. Het is dus belangrijk dat dit soort files geïnterpreteerd kunnen worden en dat de data uitgelezen kan worden. Hierdoor is het mogelijk om controles uit te gaan voeren op de data. Scriptingtalen zoals Powershell, Python en Bash zijn hiervoor plausibele oplossingen. %Interessant voor code toe te voegen hier met voorbeelden? 
 Een gewenste, maar ook belangrijke, controle binnen dit onderzoek is de controle op het formaat van het IP-adres. Hierbij moet er nagegaan worden of het IP-adres geldig is. Een geldig IP-adres is een adres die behoort tot de RC1918. 
 Wanneer dit het geval is, dan moet er nagegaan worden of het adres reeds bestaat in de firewall. Zo niet, moet het adres aangemaakt worden in de firewall door gebruik te maken van de input die door de klant werd meegegeven. Een van de voorwaarden waaraan voldaan moet worden is dat er slechts één IP-adres tegelijk doorgelaten mag worden. Hiervoor is er dus nood aan een controle op de subnetmask. Deze moet bijgevolg 255.255.255.255 bedragen om overeen te komen met een enkel ip adres \autocite{StackExchange2020}. Naast controle op het IP-adres, is er ook nood aan een controle op het aantal poorten die op hetzelfde moment mogen worden doorgelaten.  Deze zaken zullen later tijdens de uitvoer van de proof-of-concept in de praktijk worden omgezet. Uiteraard zijn er nog andere controles mogelijk. De uitvoer van dit onderzoek toont louter een oplossing voor het implementeren van controle op input dat later gebruikt kan worden in Ansible om zaken te configureren. Dit onderzoek kan dus dienen als inspiratie om een gelijkaardig probleem aan te pakken. 



\subsection{Wat is Powershell?}
Powershell is een open-source tool gemaakt door Microsoft dat standaard geïntegreerd zit in het Windows besturingssysteem. Het is een tool die werkt met een command-line interface. Daarnaast kan Powershell ook gezien worden als een scripting taal, waardoor het mogelijk is om sequentieel commando's uit te voeren. Vervolgens kan Powershell ook dienen als een programeertaal. Dit alles maakt het mogelijk voor developers, IT admins en DevOps engineers om taken te gaan automatiseren en configureren gebruikmakende van code. Vervolgens zorgt de combinatie van scriptingtaal en programeertaal ervoor dat er controles kunnen uitgevoerd worden. Hierdoor is het makkelijker om JSON files te gaan interpreteren, doordat de inhoud van de file kan opgeslagen worden in een variabele, om vervolgens controles op uit te voeren die werden geschreven in Powershell a.d.h.v. bovenstaande talen.  Dit is dus een perfecte tool voor het verdere verloop van het onderzoek \autocite{BasuMallick2022}.

\section{Hoe kunnen Ansible en Powershell samenwerken?}
Het uiteindelijke doel is om Ansible en Powershell te laten samenwerken om zo tot een geheel te komen. Echter moet er nagedacht worden over welke mogelijkheden er zijn om dit te kunnen realiseren. Zo is het niet mogelijk om Ansible op een Windows host te laten draaien. Ansible is enkel te gebruiken op Linux machines  \autocite{Hat2020}. Echter is het wel mogelijk om windows machines te configureren met behulp van Ansible. \autocite{Ansible2023} . Aangezien Ansible enkel werkt op Linux machines is het uiteraard een probleem als er gebruik gemaakt wordt van Powershell voor het uitvoeren van de eerder besproken controles , aangezien Powershell niet native op een Linux machine draait. 
Hiervoor is er een makkelijke oplossing. Het is mogelijk om Powershell te installeren op het Linux besturingssysteem \autocite{MicrosoftAnsible2023}. Aangezien Ansible gebruik maakt van CLI commando's , is het dus perfect mogelijk om Ansible met Powershell te doen samenwerken. Dit door de gewenste CLI commando's simpelweg op te roepen in een Powershell script \autocite{Ansible2023}.

\subsubsection{Welke vorm van virtualisatie en waarom?}
Voor deze use case is het belangrijk dat de manier die gehanteerd wordt voor de virtualisatie van de linux machine in zekere zin dynamisch is. Het moet zeer snel en efficient opnieuw kunnen opgezet worden op exact dezelfde manier, op verschillende hosts. Hiervoor zou is vagrant een mogelijke oplossing. Vagrant is een tool die zorgt voor het automatisch opzetten van virtuele machines en de onderhoud hiervan \autocite{samandal2021}.  Dit wordt vaak gebruikt in combinatie met Ansible. Hoewel dit op het eerste zicht lijkt op een goede oplossing zal dit niet zo worden beschouwd, aangezien het moeilijk is om Ansible werkende te krijgen op een Windows host. Dit zal enkel zorgen voor onnodige overhead wanneer men dezelfde opstelling wilt gaan gebruiken op een andere host. Om dit op te lossen, zou men eerder gebruik kunnen maken van Docker containers.

\subsubsection{Waarom Docker?}
Docker is de ideale oplossing als basis voor het proof-of-concept. Dit biedt de mogelijkheid om instanties van virtuele machines dynamisch, snel en efficiënt op te zetten. Om dit mogelijk te maken, maakt Docker gebruik van containers. simpel en breed verteld, zijn dit kant-en-klare instanties van applicaties en machines. Deze worden opgesteld aan de hand van een Docker file. In deze file worden alle zaken gedefinieerd die nodig zijn bij de opstart van een container. Het is belangrijk om te begrijpen dat alles wat manueel wordt geïnstalleerd of aangemaakt op deze containers teniet worden gedaan na herstart. Dit omdat Docker steeds de gewenste configuratie zal initialiseren die in de Docker file staat bij elk opstart. Wanneer een Docker container wordt afgesloten wordt deze in principe verwijderd. Daarnaast is het mogelijk  Docker files te plaatsen op Docker Hub. Dit maakt het mogelijk dat andere gebruikers snel dezelfde opstelling kunnen ophalen en gebruiken. Het is echter wel nodig om de Docker software te installeren op de host waarop deze gebruikt wenst te worden \autocite{Docker}. Docker is op dit moment een hot topic binnen de IT wereld. Hieromtrent zijn er nog talloze zaken die onderzocht kunnen worden. Voor dit onderzoek tot een goed einde te brengen is bovenstaande informatie voldoende. 








