\chapter{\IfLanguageName{dutch}{Stand van zaken}{State of the art}}%
\label{ch:stand-van-zaken}

% Tip: Begin elk hoofdstuk met een paragraaf inleiding die beschrijft hoe
% dit hoofdstuk past binnen het geheel van de bachelorproef. Geef in het
% bijzonder aan wat de link is met het vorige en volgende hoofdstuk.

% Pas na deze inleidende paragraaf komt de eerste sectiehoofding.

Dit hoofdstuk bevat je literatuurstudie. De inhoud gaat verder op de inleiding, maar zal het onderwerp van de bachelorproef *diepgaand* uitspitten. De bedoeling is dat de lezer na lezing van dit hoofdstuk helemaal op de hoogte is van de huidige stand van zaken (state-of-the-art) in het onderzoeksdomein. Iemand die niet vertrouwd is met het onderwerp, weet nu voldoende om de rest van het verhaal te kunnen volgen, zonder dat die er nog andere informatie moet over opzoeken \autocite{Pollefliet2011}.

Je verwijst bij elke bewering die je doet, vakterm die je introduceert, enz.\ naar je bronnen. In \LaTeX{} kan dat met het commando \texttt{$\backslash${textcite\{\}}} of \texttt{$\backslash${autocite\{\}}}. Als argument van het commando geef je de ``sleutel'' van een ``record'' in een bibliografische databank in het Bib\LaTeX{}-formaat (een tekstbestand). Als je expliciet naar de auteur verwijst in de zin, gebruik je \texttt{$\backslash${}textcite\{\}}.
Soms wil je de auteur niet expliciet vernoemen, dan gebruik je \texttt{$\backslash${}autocite\{\}}. In de volgende paragraaf een voorbeeld van elk.

\textcite{Knuth1998} schreef een van de standaardwerken over sorteer- en zoekalgoritmen. Experten zijn het erover eens dat cloud computing een interessante opportuniteit vormen, zowel voor gebruikers als voor dienstverleners op vlak van informatietechnologie~\autocite{Creeger2009}.
\pagebreak
\subsection{Wat is een Network Virtual Appliance in Azure?}
Voor het verdere verloop van dit onderzoek is het zeer belangrijk om te begrijpen wat een Azure Network Virtual Appliance (NVA) juist inhoud en wat het nut hiervan is. Een NVA is een virtuele machine met een user kernel , OS etc. die worden aangeboden in Azure door verschillende grote vendors. In dit onderzoek wordt er specifiek gebruik gemaakt van een Network Virtual Appliance in de vorm van een Fortigate firewall. Fortigate is een van de grotere spelers wanneer het gaat om firewalls. Specifiek wordt er voor Fortigate gekozen omdat dit de meest voorkomende firewall is bij de klanten van delaware. Microsoft biedt deze optie aan binnen Azure zodat het makkelijker zou zijn voor een klant om de overstap te maken naar de cloud. Dit omwille van het feit dat dezelfde software kan gebruilt worden als on premise. Dit zorgt ervoor dat er geen mensen zouden moeten omgeschoold worden om gebruik te maken van andere software. Op deze manier krijgt de klant alle voordelen die Azure te bieden heeft in combinatie met de vertrouwde configuratie en software. Naast Fortigate zijn er nog andere populaire keuzes zoals Cisco, Check Point, Barracuda etc. \autocite{MicrosoftNVA} 
Microsoft biedt binnen Azure ook zelf een firewall aan , genaamd Azure firewall. Een Azure firewall is een oplossing die Microsoft aanbiedt voor het beveiligen van virtuele netwerken binnen Azure. \autocite{Cooke} Het is een volledig alleenstaande firewall, die onderdeel uitmaakt van Azures Platform as a Service. \newline \autocite{Seferlis2018} Azure firewall biedt een scala aan verschillende oplossingen voor allerlei problemen, door de aanwezigheid van de vele functies. Dit valt echter buiten de scope van dit onderzoek. Aan de werking van Azure firewall kan namelijk een heel nieuw onderzoek gewijd worden. Deze optie zal niet worden bekeken voor dit onderzoek aangezien delaware meestal niet voor deze oplossing zal kiezen omwille van de hoge kostprijs. 

\subsection*{Wat is Infrastructure-as-Code?}
Infrastructure-as-Code (IaC) zal tijdens de opbouw van het proof-of-concept een belangrijke rol spelen voor de automatisatie en configuratie van de firewall. Infrastructure-as-Code is het bouwen en onderhouden van infrastructuur aan de hand van code. Dit is een technologie die ontstaan is door de opmars van de cloud. Vroeger was code enkel voor software en was er hardware nodig voor de infrastructuur. Veel zaken in de cloud zijn nu virtueel. Dat zorgt ervoor dat IaC mogelijk is. Voor het bouwen van de infrastructuren wordt er gebruik gemaakt van templates die de infrastructuur omschrijven als een object. Deze objecten worden omschreven in YML-, JSON- of XML-formaat. Zo biedt het ook enorm veel voordelen. Het interessantste voordeel voor dit onderzoek is dat, door middel van IaC, de infrastructuur zeer makkelijk uit te rollen valt naar meerdere omgevingen. \autocite{Bulthuis} Het biedt ook enorm veel mogelijkheden op vlak van uitbreiding en flexibiliteit. Het is steeds mogelijk je infrastructuur zeer snel aan te passen en uit te breiden. \autocite{Morris2016}  Voor dit onderzoek wordt er specifiek gezocht naar een oplossing in Azure met potentieel verschillende vormen firewalls. 
Hiervoor zijn verschillende oplossingen. Terraform kan worden gebruikt voor het managen en configureren van servers binnen Azure, AWS en Google Cloud. \autocite{IONOS2019} \autocite{Janashia2020} Terraform bevat wel enkele beperkingen die een groot struikelblok zijn voor de use case van dit onderzoek. --TODO--

Een andere nuttige tool voor de gewenste use case is Ansible. Dit is een open source tool die het mogelijk maakt de configuratie van virtuele machines ,zowel on premise als in de cloud , te automatiseren. \autocite{Hat} Ansible biedt een catalogus aan van verschillende 



