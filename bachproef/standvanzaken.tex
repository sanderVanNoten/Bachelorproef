\chapter{\IfLanguageName{dutch}{Stand van zaken}{State of the art}}%
\label{ch:stand-van-zaken}

In dit hoofdstuk wordt er gekeken naar alle informatie die er nodig is voor het uitvoeren van het onderzoek. Dit zal gebeuren aan de hand van een relevante literatuurstudie in combinatie met een kort interview met belanghebbende voor het opstellen van een korte requirements-analyse. Dit zorgt voor een gestructureerde manier van werken. Die bijgevolg zorgt voor een makkelijkere opbouw van het proof-of-concept. \newline

De zaken die besproken zullen worden zijn belangrijk voor het verdere verloop van het onderzoek. Als eerst wordt er gekeken naar wat een Network Virtual Appliance juist inhoudt en wat het nut hiervan is. Vervolgens wordt er wat dieper ingegaan op het begrip Infrastructure As Code en welke tools hiertoe behoren. Verder zal er al kort worden beschreven welke controles er nodig zijn en hoe deze kunnen worden toegepast. Na een korte omschrijving van welke tools er beschikbaar zijn voor het oplossen van het probleem , wordt er specifiek gekeken naar wat deze tools juist zijn en hoe ze zullen worden gebruikt. Vervolgens wordt er gekeken naar hoe deze tools het best kunnen samenwerken en welke problemen dit met zich meebrengt. Verder wordt er gekeken naar virualisatie en welke vorm het best geschikt is voor het onderzoek.
% Tip: Begin elk hoofdstuk met een paragraaf inleiding die beschrijft hoe
% dit hoofdstuk past binnen het geheel van de bachelorproef. Geef in het
% bijzonder aan wat de link is met het vorige en volgende hoofdstuk.

% Pas na deze inleidende paragraaf komt de eerste sectiehoofding.

% Dit hoofdstuk bevat je literatuurstudie. De inhoud gaat verder op de inleiding, maar zal het onderwerp van de bachelorproef *diepgaand* uitspitten. De bedoeling is dat de lezer na lezing van dit hoofdstuk helemaal op de hoogte is van de huidige stand van zaken (state-of-the-art) in het onderzoeksdomein. Iemand die niet vertrouwd is met het onderwerp, weet nu voldoende om de rest van het verhaal te kunnen volgen, zonder dat die er nog andere informatie moet over opzoeken \autocite{Pollefliet2011}.

% Je verwijst bij elke bewering die je doet, vakterm die je introduceert, enz.\ naar je bronnen. In \LaTeX{} kan dat met het commando \texttt{$\backslash${textcite\{\}}} of \texttt{$\backslash${autocite\{\}}}. Als argument van het commando geef je de ``sleutel'' van een ``record'' in een bibliografische databank in het Bib\LaTeX{}-formaat (een tekstbestand). Als je expliciet naar de auteur verwijst in de zin, gebruik je \texttt{$\backslash${}textcite\{\}}.
% Soms wil je de auteur niet expliciet vernoemen, dan gebruik je \texttt{$\backslash${}autocite\{\}}. In de volgende paragraaf een voorbeeld van elk.

% \textcite{Knuth1998} schreef een van de standaardwerken over sorteer- en zoekalgoritmen. Experten zijn het erover eens dat cloud computing een interessante opportuniteit vormen, zowel voor gebruikers als voor dienstverleners op vlak van informatietechnologie~\autocite{Creeger2009}.
% \pagebreak
\newpage
\section{Wat is een Network Virtual Appliance in Azure?}
Voor het verdere verloop van dit onderzoek is het zeer belangrijk om te begrijpen wat een Azure Network Virtual Appliance (NVA) juist inhoudt en wat het nut hiervan is. Een NVA is een virtuele machine met een user kernel , OS etc. die worden aangeboden in Azure door verschillende grote vendors. In dit onderzoek wordt er specifiek gebruik gemaakt van een Network Virtual Appliance in de vorm van een Fortigate firewall. Fortigate is een van de grotere spelers wanneer het gaat om firewalls. Specifiek wordt er voor Fortigate gekozen omdat dit de meest voorkomende firewall is bij de klanten van delaware. Microsoft biedt deze optie aan binnen Azure zodat het makkelijker zou zijn voor een klant om de overstap te maken naar de cloud. Dit omwille van het feit dat dezelfde software kan gebruikt worden als on premise. Dit zorgt ervoor dat er geen mensen zouden moeten omgeschoold worden om gebruik te maken van andere software. Op deze manier krijgt de klant alle voordelen die Azure te bieden heeft in combinatie met de vertrouwde configuratie en software. Naast Fortigate zijn er nog andere populaire keuzes zoals Cisco, Check Point, Barracuda etc. \autocite{MicrosoftNVA} 
Microsoft biedt binnen Azure ook zelf een firewall aan , genaamd Azure firewall. Een Azure firewall is een oplossing die Microsoft aanbiedt voor het beveiligen van virtuele netwerken binnen Azure. \autocite{Cooke} Het is een volledig alleenstaande firewall, die onderdeel uitmaakt van Azures Platform as a Service. \newline \autocite{Seferlis2018} Azure firewall biedt een scala aan verschillende oplossingen voor allerlei problemen, door de aanwezigheid van de vele functies. De werking van een Azure firewall valt echter buiten de scope van dit onderzoek. Aan de werking van Azure firewall kan namelijk een heel nieuw onderzoek gewijd worden. Deze optie zal niet worden bekeken voor dit onderzoek aangezien delaware meestal niet voor deze oplossing zal kiezen omwille van de hoge kostprijs. 

\subsection*{Wat is Infrastructure-as-Code?}
Infrastructure-as-Code (IaC) zal tijdens de opbouw van het proof-of-concept een belangrijke rol spelen voor de automatisatie en configuratie van de firewall. Infrastructure-as-Code is het bouwen en onderhouden van infrastructuur aan de hand van code. Dit is een technologie die ontstaan is door de opmars van de cloud. Vroeger was code enkel voor software en was er hardware nodig voor de infrastructuur. Veel zaken in de cloud zijn nu virtueel. Dat zorgt ervoor dat IaC mogelijk is. Voor het bouwen van de infrastructuren wordt er gebruik gemaakt van templates die de infrastructuur omschrijven als een object. Deze objecten worden omschreven in YML-, JSON- of XML-formaat. Zo biedt het ook enorm veel voordelen. Het interessantste voordeel voor dit onderzoek is dat, door middel van IaC, de infrastructuur zeer makkelijk uit te rollen valt naar meerdere omgevingen. \autocite{Bulthuis} Het biedt ook enorm veel mogelijkheden op vlak van uitbreiding en flexibiliteit. Het is steeds mogelijk je infrastructuur zeer snel aan te passen en uit te breiden. \autocite{Morris2016}  Voor dit onderzoek wordt er specifiek gezocht naar een oplossing in Azure met potentieel verschillende vormen firewalls. 
Hiervoor zijn verschillende oplossingen. Terraform kan worden gebruikt voor het managen en configureren van servers binnen Azure, AWS en Google Cloud. \autocite{IONOS2019} \autocite{Janashia2020} Terraform bevat wel enkele beperkingen die een groot struikelblok zijn voor de use case van dit onderzoek. Terraform is namelijk een zeer goede oplossing voor use cases die volledig in IaC worde geimplementeerd. Bij de use case van dit onderzoek daarentegen zullen er nog steeds zaken manueel worden geconfigureerd op de firewall. Dit zal dus niet volledig in IaC verlopen. Dit omwille van het feit dat Terraform met een state file werkt. Deze state file houdt alle configuratie die gemaakt is in IaC bij. Wanneer iemand bijgevolg manueel een aanpassing maakt op de firewall , zal na het uitvoeren van de Terraform code ervoor zorgen dat deze manuele wijziging overschreven wordt. Dit is dus een struikelblok. Een andere nuttige tool voor de gewenste use case is Ansible.

\subsubsection{Wat is Ansible?}
Dit is een open source tool die het mogelijk maakt de configuratie van virtuele machines ,zowel on premise als in de cloud , te automatiseren. \autocite{Hat} Ansible biedt een catalogus aan van verschillende modules die kunnen gebruikt worden voor allerlei doeleinde. Zo is er een module voor het configureren van Fortigate firewalls. Dit zal dus in rekening worden genomen bij het maken van het proof-of-concept. \autocite{Fortinet2020} Ansible lost daarbij ook het probleem op dat Terraform heeft. Ansible zal enkel de gewenste zaken doorvoeren, ongeacht of deze al bestaat ja of nee. Wanneer deze al bestaat wordt deze dus niet overschreden. Wat wel het geval zal zijn bij het gebruik van Terraform. Ansible lijkt bijgevolg de meest vatbare oplossing voor het configureren van de firewall. 

\subsection{Controles?}

Zoals eerder besproken in de probleemstelling , is het de bedoeling dat een klant aan de hand van een webportaal firewall regels automatisch kan doorgeven aan de firewall. Het probleem dat delaware heeft bij deze use case is dat ze graag controle zouden kunnen uitvoeren op de data die de klant meegeeft. Eerder is Ansible al als een oplossing naar boven gekomen voor het configureren van de firewall. Echter is er geen manier om controle toe te passen op input die de klant zal meegeven met Ansible. Er is dus nood aan een alternatieve extra stap. De input van de klant zal geformateerd worden naar JSON formaat. Het is dus belangrijk dat dit soort files geinterpreteerd kunnen worden en de data uitgelezen kan worden. Op deze manier wordt het mogelijk om controles uit te gaan voeren op deze data. Scriptingtalen zoals powershell, python en bash zijn hiervoor bijvoorbeeld een plausibele oplossing. %Interessant voor code toe te voegen hier met voorbeelden? 
Controle op het formaat van het ip adres is bijvoorbeeld een belangrijke gewenste controle. Is het een geldig ip adres? zoja , kijk of het al reeds bestaat in de firewwall. Zo niet, maak dit adres aan in de firewall gebruikmakend van de input van de klant.  Er mag bijvoorbeeld steeds maar een enkel ip adres tegelijk doorgelaten worden. Hiervoor is er dus nood aan een controle op de subnetmask. Deze moet bijgevolg 255.255.255.255 bedragen om overeen te komen met een enkel ip adres. \autocite{StackExchange2020} Vervolgens is er ook bijvoorbeeld nood aan controle op het aantal poorten die tegelijkertijd mogen worden doorgelaten. Deze zaken zullen later tijdens de uitvoer van het proof-of-concept in de praktijk worden omgezet. Uiteraard zijn er nog andere controles mogelijk. De uitvoer van dit onderzoek toont louter een oplossing voor het implementeren van controle op input dat later gebruikt kan worden in Ansible om zaken te configureren. Dit onderzoek kan dus dienen als inspiratie om een gelijkaardig probleem aan te pakken. 



\subsection{Wat is powershell?}

Powershell is een open-source tool gemaakt door Microsoft dat standaard geintegreerd zit in het Windows besturingssysteem. Het is een tool die werkt met een command-line interface. Powershell maakt het mogelijk voor developers, IT admins en DevOps engineers om taken te gaan automatiseren en configureren gebruikmakend van code. Powershell is een scripting taal die het mogelijk maakt om sequentieel commando's uit te voeren. Daarnaast kan het ook dienen als programmeertaal. De combinatie van deze twee zorgt ervoor dat je controles kan uitvoeren. Zo maakt dit het mogelijk om JSON files te interpreteren, deze inhoud op te slaan in een variabele om vervolgens op deze data controle toe te passen. Dit is dus een perfecte tool voor het verdere verloop van het onderzoek.  \autocite{BasuMallick2022}

\subsection{Hoe kunnen Ansible en Powershell samenwerken?}

Het uiteindelijke doel is om beide tools te doen samenwerken om op deze manier een geheel te bekomen. Hiervoor moet er nagedacht worden over welke mogelijkheden er zijn om dit te kunnen realiseren. Zo is het niet mogelijk om Ansible op een Windows host te draaien. Je kan enkel Windows hosts configureren aan de hand van Ansible. Het is echter niet mogelijk om dit vanuit een Windows machine te doen, hoewel hier work arounds voor bestaan . \autocite{Ansible2023} Dit is uiteraard een probleem als er gebruik gemaakt wordt van Powershell voor het uitvoeren van de eerder besproken controles. Het is enkel mogelijk om Ansible native te draaien op een linux machine. \autocite{Hat2020} Het zal dus noodzakelijk zijn om gebruik te maken van virtualisatie voor het mogelijk maken van het proof-of-concept. Voor het probleem omtrent Powershell is er een makkelijke oplossing. Het is namelijk mogelijk om powershell te installeren op het Linux besturingssysteem. \autocite{MicrosoftAnsible2023} 

\subsubsection{Welke vorm van virtualisatie en waarom?}
Voor deze use case is het belangrijk dat de manier die gehanteerd wordt voor de virtualisatie van de linux machine in zekere zin dynamisch is. Het moet zeer snel en efficient opnieuw kunnen opgezet worden op exact dezelfde manier, op verschillende hosts. Hiervoor zou is vagrant een mogelijke oplossing. Vagrant is een tool die zorgt voor het automatisch opzetten van virtuele machines en het onderhoudt hiervan. \autocite{samandal2021} Dit wordt vaak gebruikt in combinatie met Ansible. Hoewel dit op het eerste zicht lijkt op een goede oplossing zal dit niet zo worden beschouwd, aangezien het moeilijk is om ansible werkende te krijgen op een Windows host. Dit zal enkel zorgen voor onnodige overhead wanneer men dezelfde opstelling wilt gaan gebruiken op een andere host. Een andere oplossing voor dit probleem zijn Docker containers. 

\subsubsection{Waarom docker?}
Docker is de ideale oplossing als basis voor het proof-of-concept. Dit biedt de mogelijkheid dynamisch , snel en efficient instanties op te zetten van virtuele machines. Om dit mogelijk te maken, maakt docker gebruik van containers. Heel simpel en breed verteld, zijn dit kant en klare instanties van applicaties en machines. Deze worden opgesteld aan de hand van een docker file. In deze file worden alle zaken die geinstalleerd moeten worden bij opstart van de container gedefinieerd. Het is belangrijk om te begrijpen dat alles wat manueel wordt geinstalleerd of aangemaakt op deze containers teniet worden gedaan na restart. Daarom is het belangrijk om in deze docker file alle benodigde zaken te definieren. Daarnaast is het mogelijk  docker files te plaatsen op Docker Hub. Dit maakt het mogelijk dat andere gebruikers snel dezelfde opstelling kunnen ophalen en gebruiken. In dit onderzoek zal er eerder geopteerd worden voor de docker file door te geven. Wat ook perfect mogelijk is, iedereen kan deze opzetten. Het is echter wel nodig om de Docker software te installeren op de host waarop deze gebruikt wenst te worden. \autocite{Docker} 






